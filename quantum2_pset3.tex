 
\documentclass[12pt]{article}
\usepackage{a4wide}
\usepackage{color, amssymb}
\usepackage[margin=1in]{geometry}
\usepackage[document]{ragged2e}
\usepackage[table]{xcolor}
\usepackage{multirow}
\usepackage[braket, qm]{qcircuit}
\setlength{\arrayrulewidth}{0.5mm}
\setlength{\tabcolsep}{16pt}
\renewcommand{\arraystretch}{1.9}
\usepackage[english,greek]{babel}
\usepackage{braket}
\usepackage{mathtools}
\usepackage{ragged2e}
\renewcommand{\baselinestretch}{1.5}
\input{epsf}
\usepackage{float}
\usepackage{graphicx}
\usepackage{caption}
\usepackage{subcaption}
\usepackage{cancel}
\usepackage{algorithm}
\usepackage{animate}
\usepackage[noend]{algpseudocode}

\begin{document}

\greektext

\noindent\rule{\textwidth}{2pt}
\begin{center}
{\bf ΚΒΑΝΤΙΚΗ ΤΕΧΝΟΛΟΓΙΑ}\\ 
{\bf 3o Σετ Ασκήσεων }\\
{\bf Καλαμαράκης Θεόδωρος:} 2018030022\\
\end{center}
\rule{\textwidth}{.5pt}
\noindent

\begin{center}

\end{center}
 
 

\justifying

%%%%%%%%%%%%%%%%%%%%%%%%%%%%%%%%%%%%%%%%%%%%%%%%%%%%%%%%%%%%%%%%%%%%%%%%%%%%%%%%%%%%%%%%%%%%%%%%%%%%%%%%%%%%%%%%%%%%%%%%%%%%%%%%%%%%%%%
\section*{{\bfΑπόδειξη της σχέσης $6.7$}}
    Απο την σχέση $(6.6)$ έχουμε :
        \begin{align*}
            H &= \sqrt{8E_cE_J}\left(\hat{c}^\dag \hat{c}+\frac{1}{2}\right) - E_J - \frac{E_c}{12}\left(\hat{c}^\dag+\hat{c}\right)^4   
        \end{align*}
        Αναπτύσσουμε το όρο $\left(\hat{c}^\dag+\hat{c}\right)^4$  κρατώντας μόνο τους όρους με ίσο αριθμό $\hat{c}^\dag$ και $\hat{c}$
        \begin{align*}\left(\hat{c}^\dag+\hat{c}\right)^4 \approx \left(\hat{c}^\dag\right)^2\hat{c}^2 + \left(\hat{c}^\dag\hat{c}\right)^2 + \hat{c}^\dag\hat{c}^2\hat{c}^\dag +  \hat{c}\left(\hat{c}^\dag \right)^2\hat{c}+ \left(\hat{c}\hat{c}^\dag\right)^2+\hat{c}^2\left(\hat{c}^\dag\right)^2 \tag{1}\end{align*}
        Γνωρίζουμε οτι $\hat{c} = \sum_j\sqrt{j+1}\ket{j}\bra{j+1}$ kai $\hat{c}^\dag = \sum_j\sqrt{j+1}\ket{j+1}\bra{j}$ 
        Αρα 
        $$
 \left\{
\begin{array}{lr}
    \hat{c}^\dag\hat{c} = \sum_j(j+1)\ket{j+1}\bra{j+1}\\
    \hat{c}\hat{c}^\dag = \sum_j(j+1)\ket{j}\bra{j}
\end{array}
\right\} \Leftrightarrow 
\left\{
    \begin{array}{lr}
        \hat{c}^\dag\hat{c} = \sum_jj\ket{j}\bra{j}\\
        \hat{c}\hat{c}^\dag = \sum_j(j+1)\ket{j}\bra{j}
    \end{array}
    \right\} \Rightarrow 
$$
$$\Rightarrow \hat{c}\hat{c}^\dag - \hat{c}^\dag\hat{c} = \sum_j(j+1)\ket{j}\bra{j} - \sum_jj\ket{j}\bra{j} = \sum_j\ket{j}\bra{j} = I$$ $$ \Leftrightarrow \hat{c}\hat{c}^\dag = \hat{c}^\dag\hat{c} +I $$
Ξαναγράφουμε τους όρους τις $(1)$ με βάση την τελευταια εξίσωση
\begin{itemize}
    \item $$\left(\hat{c}^\dag\right)^2\hat{c}^2 = \hat{c}^\dag \left(\hat{c}\hat{c}^\dag -I\right)\hat{c} =  \left(\hat{c}^\dag\hat{c}\right)^2 -\hat{c}^\dag\hat{c}$$
    \item $$\hat{c}^\dag\hat{c}^2\hat{c}^\dag  = \hat{c}^\dag \hat{c}\left(\hat{c}^\dag\hat{c} +I\right) =  \left(\hat{c}^\dag\hat{c}\right)^2 +\hat{c}^\dag\hat{c}$$
    \item $$\hat{c}\left(\hat{c}^\dag \right)^2\hat{c}  =\left(\hat{c}^\dag\hat{c} +I\right) \hat{c}^\dag \hat{c} =  \left(\hat{c}^\dag\hat{c}\right)^2 +\hat{c}^\dag\hat{c}$$
    \item $$\left(\hat{c}\hat{c}^\dag\right)^2  =\left(\hat{c}^\dag\hat{c} +I\right)^2  =  \left(\hat{c}^\dag\hat{c}\right)^2 +2\hat{c}^\dag\hat{c}+I$$
    \item $$\hat{c}^2\left(\hat{c}^\dag\right)^2  =\hat{c} \left(\hat{c}^\dag\hat{c} +I\right)\hat{c}^\dag =  \left(\hat{c}\hat{c}^\dag\right)^2 +\hat{c}\hat{c}^\dag= \left(\hat{c}^\dag\hat{c}\right)^2 +2\hat{c}^\dag\hat{c}+I +\hat{c}^\dag\hat{c} +I = \left(\hat{c}^\dag\hat{c}\right)^2 +3\hat{c}^\dag\hat{c} +2I$$
\end{itemize}
Αρα η $(1)$ γράφεταο 
\begin{align*}\left(\hat{c}^\dag+\hat{c}\right)^4 \approx  6\left(\hat{c}^\dag\hat{c}\right)^2 +6\hat{c}^\dag\hat{c} +3Ι\end{align*}
Συνεπώς 
\begin{align*}
    H &= \sqrt{8E_cE_J}\left(\hat{c}^\dag \hat{c}+\frac{1}{2}\right) - E_J - \frac{E_c}{12}\left(\hat{c}^\dag+\hat{c}\right)^4  =\\
    &= \sqrt{8E_cE_J}\left(\hat{c}^\dag \hat{c}+\frac{1}{2}\right) - E_J - \frac{E_c}{12}\left(6\left(\hat{c}^\dag\hat{c}\right)^2 +6\hat{c}^\dag\hat{c} +3Ι\right)=\\
    &= \sqrt{8E_cE_J}\left(\hat{c}^\dag \hat{c}+\frac{1}{2}\right) - E_J - \frac{E_c}{2}\left(\left(\hat{c}^\dag\hat{c}\right)^2 +\hat{c}^\dag\hat{c} +\frac{1}{2}Ι\right)
\end{align*}
        Αγνοώντας τις σταθερές και θέτοντας $\omega_0 = \sqrt{8E_cE_J}$ και $\delta = -E_c$ έχουμε
        \begin{align*}
            H&=\omega_0\hat{c}^\dag \hat{c} +\frac{\delta}{2}\left(\left(\hat{c}^\dag\hat{c}\right)^2 +\hat{c}^\dag\hat{c}\right) = \left(\omega_0 +\frac{\delta}{2}\right)\hat{c}^\dag\hat{c} + \frac{\delta}{2}\left(\hat{c}^\dag\hat{c}\right)^2
        \end{align*}
\rule{\textwidth}{.5pt}
\section*{{\bfΑπόδειξη της σχέσης $7.5$}}

\begin{align*}
    U\sigma^{\pm}U^\dag &= \left(I\cos(\omega_q t/2) - i\sigma^z\sin(\omega_q t/2)\right)\sigma^\pm\left(I\cos(\omega_q t/2) + i\sigma^z\sin(\omega_q t/2)\right) = \\
    & = \cos^2(\omega_q t/2)I\sigma^\pm I + \cos(\omega_q t/2)\sin(\omega_q t/2)iI\sigma^\pm\sigma^z\\
    & - \cos(\omega_q t/2)\sin(\omega_q t/2)i\sigma^z\sigma^\pm I+\sin^2(\omega_q t/2)\sigma^z\sigma^\pm\sigma^z \stackrel{(*)}{ = }\\
    &\stackrel{(*)}{ = }\cos^2(\omega_q t/2)\sigma^\pm  \pm \cos(\omega_q t/2)\sin(\omega_q t/2)i\sigma^\pm\\
    & \pm \cos(\omega_q t/2)\sin(\omega_q t/2)i\sigma^\pm -\sin^2(\omega_q t/2)\sigma^\pm =\\
    & = \sigma^\pm \left(\cos^2(\omega_q t/2) \pm 2i\cos(\omega_q t/2)\sin(\omega_q t/2)-\sin^2(\omega_q t/2) \right) \stackrel{(**)}{=}\\
    &\stackrel{(**)}{=}\sigma^\pm \left(\cos(\omega_q t)\pm i\sin(\omega t)\right) =\sigma^\pm e^{\pm i\omega_q t}
\end{align*}

Όπου στο σημείο $(*)$ χρησιμοποιούμε οτι 
$$\sigma^\pm \sigma^z=-\sigma^z\sigma^\pm = \pm \sigma^\pm \textnormal{    και    }  \sigma^z \sigma^\pm \sigma^z =  - \sigma^z \sigma^z  \sigma^\pm = -I\sigma^\pm = -\sigma^\pm $$
Αφού ο $\sigma^z$ είναι \textlatin{Hermitian} και \textlatin{Unitary}\\
Στο σημειό $(**)$ χρησιμοποιούμε οτι $$\cos \theta = \cos^2 \frac{\theta}{2} - \sin^2 \frac{\theta}{2} \textnormal{   και   }  \sin \theta = 2\cos \frac{\theta}{2}\sin \frac{\theta}{2}$$
\rule{\textwidth}{.5pt}
\section*{{\bfΑπόδειξη της σχέσης $7.9$}}

Απο τη σχέση $7.6$ έχουμε οτι:
$$\hat{H}_{d,I}^{(RWA)} = -\hbar\left(\Omega e^{-i\Delta_q t} + \tilde{\Omega}e^{i(\omega_q + \omega_d)t}\right)\sigma^+ - \hbar\left(\Omega^* e^{i\Delta_q t} + \tilde{\Omega}^*e^{-i(\omega_q + \omega_d)t}\right)\sigma^-$$
Αγνοώντας τους όρους $\omega_q +\omega_d$ η σχέση γίνεται: 
$$\hat{H}_{d,I}^{(RWA)} = -\hbar\Omega e^{-i\Delta_q t}\sigma^+ - \hbar\Omega^* e^{i\Delta_q t}\sigma^- $$
Aρα 
\begin{align*}
    \hat{H}_{d}^{(RWA)} &= U\hat{H}_{d,I}^{(RWA)}U^\dag = -\hbar\Omega e^{-i\Delta_q t}U\sigma^+U^\dag - \hbar\Omega^* e^{i\Delta_q t}U\sigma^-U^\dag \stackrel{(7.5)}{=}\\
    &\stackrel{(7.5)}{=} -\hbar\Omega e^{-i\Delta_q t}e^{ i\omega_q t}\sigma^+  -\hbar\Omega^* e^{i\Delta_q t}e^{- i\omega_q t}\sigma^- =  \\
    &=  -\hbar\Omega e^{i(-\omega_q + \omega_d+\omega_q) t}\sigma^+  -\hbar\Omega^*e^{i(\omega_q - \omega_d-\omega_q) t}\sigma^-=\\
    &= -\hbar\Omega e^{+i \omega_d t}\sigma^+  -\hbar\Omega^* e^{-i \omega_d t}\sigma^-
\end{align*}
Άρα
$$\hat{H}^{(RWA)}=\hat{H}_{0}+ \hat{H}_{d}^{(RWA)} = -\frac{1}{2}\hbar \omega_q\sigma^z -\hbar\Omega e^{i \omega_d t}\sigma^+  -\hbar\Omega^* e^{-i \omega_d t}\sigma^-$$
\rule{\textwidth}{.5pt}
\section*{{\bfΑπόδειξη της σχέσης $7.12$}}
Aπο την $(7.10)$ έχουμε
\begin{align*}
    \hat{H}_{eff} &= U_d\hat{H}^{RWA}U_d^\dag - i\hbar U_d \dot{U}^\dag_d \stackrel{(7.9)}{=} \\
    &=-\frac{1}{2}\hbar \omega_qU_d\sigma^zU_d^\dag -\hbar\Omega e^{i \omega_d t}U_d\sigma^+U_d^\dag  -\hbar\Omega^* e^{-i \omega_d t}U_d\sigma^-U_d^\dag - i\hbar U_d \dot{U}^\dag_d
\end{align*}
Δουλεύοντας το κάθε όρο ξεχωριστα:
\begin{itemize}
    \item Για τον όρο $\frac{1}{2}\hbar \omega_qU_d\sigma^zU_d^\dag$
        \begin{align*}
            U_d\sigma^zU_d^\dag&=e^{-i\omega_d t \sigma^z/2}\sigma^z e^{i\omega_d t \sigma^z/2} \stackrel{(*)}{=}\\
            &= \left(e^{-i\omega_d t/2}\ket{0}\bra{0}+e^{i\omega_d t/2}\ket{1}\bra{1}\right)\left(\ket{0}\bra{0}-\ket{1}\bra{1}\right)\left(e^{i\omega_d t/2}\ket{0}\bra{0}+e^{-i\omega_d t/2}\ket{1}\bra{1}\right)=\\
            &=\left(e^{-i\omega_d t/2}\ket{0}\bra{0}-e^{i\omega_d t/2}\ket{1}\bra{1}\right)\left(e^{i\omega_d t/2}\ket{0}\bra{0}+e^{-i\omega_d t/2}\ket{1}\bra{1}\right)=\\
            &=e^{-i\omega_d t/2}e^{i\omega_d t/2}\ket{0}\bra{0}-e^{i\omega_d t/2}e^{-i\omega_d t/2}\ket{1}\bra{1} = \ket{0}\bra{0} - \ket{1}\bra{1} =\\
            &= \sigma^z
        \end{align*}

    Όπου στο σημείο $(*)$ χρησιμοποιήσαμε οτι 
    $$\sigma^z = \ket{0}\bra{0}-\ket{1}\bra{1} \textnormal{   και   } e^{-i\omega_d t \sigma^z/2} = e^{-i\omega_d t (+1)/2}\ket{0}\bra{0}+  e^{-i\omega_d t (-1)/2}\ket{1}\bra{1}$$
    Αρα $\frac{1}{2}\hbar \omega_qU_d\sigma^zU_d^\dag = \frac{1}{2}\hbar \omega_q\sigma^z$
    \item Για τον όρο $\hbar\Omega e^{i \omega_d t}U_d\sigma^+U_d^\dag$
        $$\hbar\Omega e^{i \omega_d t}U_d\sigma^+U_d^\dag = \hbar\Omega e^{i \omega_d t}e^{-i\omega_d t \sigma^z/2}\sigma^+e^{i\omega_d t \sigma^z/2}$$
        Απο την σχέση $(3.23)$ της διάλεξης \textlatin{Spin qubits and Rabi oscillations} έχουμε:
        $$\hbar\Omega e^{i \omega_d t}e^{-i\omega_d t \sigma^z/2}\sigma^+e^{i\omega_d t \sigma^z/2} = \hbar\Omega e^{i \omega_d t}e^{-i\omega_d t }\sigma^+ = \hbar\Omega \sigma^+$$
    \item Για τον όρο $\hbar\Omega^* e^{-i \omega_d t}U_d\sigma^-U_d^\dag $
        \begin{align*}
            \hbar\Omega^* e^{-i \omega_d t}U_d\sigma^-U_d^\dag &=  \hbar\Omega^* e^{-i \omega_d t}e^{-i\omega_d t \sigma^z/2}\sigma^-e^{i\omega_d t \sigma^z/2}\stackrel{(3.23)}{=}\\
            &=\hbar\Omega^* e^{-i \omega_d t}e^{i\omega_d t }\sigma^- = \hbar\Omega^* \sigma^-
        \end{align*}
    \item Για τον όρο $i\hbar U_d \dot{U}^\dag_d $
        \begin{align*}
            i\hbar U_d \dot{U}^\dag_d &=  i\hbar e^{-i\omega_d t \sigma^z/2}\frac{d}{dt}e^{i\omega_d t \sigma^z/2}=-\frac{1}{2}\hbar \omega_d\left(e^{-i\omega_d t \sigma^z/2}\sigma^ze^{i\omega_d t \sigma^z/2}\right) \stackrel{(!)}{=}\\
            &=-\frac{1}{2}\hbar \omega_d\sigma^z
        \end{align*}
        Όπου στο σημείο $(!)$ το γινόμενο μέσα στην παρένθεση έχει υπολογιστεί στον πρώτο όρο 
    \end{itemize}

    Άρα 
    \begin{align*}
        \hat{H}_{eff}&= -\frac{1}{2}\hbar \omega_q\sigma^z - \hbar\Omega \sigma^+ -\hbar\Omega^* \sigma^- + \frac{1}{2}\hbar \omega_d\sigma^z =\\
        &=-\frac{1}{2}\hbar \Delta_q\sigma^z - \hbar\Omega \sigma^+ -\hbar\Omega^* \sigma^-
    \end{align*}
    Χρησιμοποιώντας την $(3.21)$ απο \textlatin{Spin qubits and Rabi oscillations} και θεωρώντας οτι $\Omega=\Omega^*$ παίρνουμε
    \begin{align*}
        \hat{H}_{eff}&= -\frac{1}{2}\hbar \Delta_q\sigma^z - \hbar\Omega \sigma^+ -\hbar\Omega^* \sigma^- = -\frac{1}{2}\hbar \Delta_q\sigma^z - \hbar\Omega (\sigma^+ + \sigma^-) =\\
        &= -\frac{1}{2}\hbar \Delta_q\sigma^z - \hbar\Omega \sigma^x
    \end{align*}
%%%%%%%%%%%%%%%%%%%%%%%%%%%%%%%%%%%%%%%%%%%%%%%%%%%%%%%%%%%%%%%%%%%%%%%%%%%%%%%%%%%%%%%%%%%%%%%%%%%%%%%%%%%%%%%%%%%%%%%%%%%%%%%%%%%%%%%

\end{document}